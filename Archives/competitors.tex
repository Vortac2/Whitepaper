\section{Competitors}

\subsection{Golem Network}

Golem grand vision is about a global, open sourced, decentralized supercomputer that anyone can access. It's made up of the combined power of user's machines, from personal laptops to entire datacenters. Anyone will be able to use Golem to compute (almost) any program, from rendering to research to running websites, in a completely decentralized and inexpensive way. The Golem Network would like to achieve a decentralized sharing economy of computing power, where anyone can make money 'renting' out their computing power or developing and selling software. [35]\\

Golem rewards are done through Ethereum tokens. Developers are required to develop computational tasks following a certain API, so that tasks can be distributed to users willing to compute them. Developers purchase Golem tokens and give them to users who are calculating for them.\\

At time of writing, Golem implements a distributed CGI rendering prototype as showcase of Golem abilities. However, Distributed Rendering is not something difficult to implement. BURP is a BOINC project that implements it [36]. The Global Processing Unit project also implements a distributed rendering mechanism among many other functionalities [37].\\


\subsection{Science Power and Research Coin (SPARC)}

SPARC is an Ethereum token which will soon experience its own Initial Coin Offering (ICO).\\

Like Gridcoin, SPARC is initially leveraging existing demand and infrastructure.
% The Berkeley Open Infrastructure for
%Network Computing (BOINC) is a computing platform comprised of 1,128,897 compute nodes. At
%17.1 petaFLOPS, it is the third most powerful supercomputer on Earth and supports computing
%projects in astrophysics, mathematics, and biology, for example. 
The alpha version of SPARC network
connects to the BOINC network and rewards participant nodes with SPARC coins for computational work performed. Researchers and developers requiring
computing power purchase SPARC tokens from an exchange and attach them to their projects.
These tokens are distributed to the participant nodes in proportion to work performed. SPARC
tokens can then be exchanged directly for computing power from the network or traded for
conventional currency on an exchange. [34]\\

In short at time of writing, SPARC is rewarding BOINC work with premined tokens through a centralized website. By contrast, Gridcoin is rewarding researchers on a dedicated blockchain in a decentralized manner.\\


\subsection{Einsteinium}

Einsteinium (shorted with EMC2) is a Bitcoin-like currency with a philanthropic side goal of funding scientific research. It lets community members vote on which worthwhile scientific research projects the proceeds should be awarded. The coin was launched on March 1st 2017. [33]\\

%"Scientific research it is [sic] a long-term investment in our future, and the future of our planet," the foundation claims. %"Funding around the world for the ‘big ideas’ has fallen dramatically in recent years.
%there is no restriction on eligible projects other than that the "science involved is pushing our understanding forward and %could build us a better, safer future.
%"To reach a wider audience than just the mining community it is essential that Einsteinium is freely traded on the exchanges and used for purchasing goods and services," the foundation writes. "Einsteinium should be as available as possible to as many people as possible to enable that trading."

EMC2 automatically donates 2\% of every block mined to the Foundation Fund to be used for donations. The mining of Einsteinium is divided into Epochs: each Epoch mines 36000 blocks of coins and is targeted to last approximately 25 days. Every 25 days, at the end of each Epoch, a new ground breaking scientific cause is selected to receive Einsteinium Foundation funding. [33]\\

Like Bitcoin, Einsteinium is a distributed peer-2-peer digital currency released without any premine. EMC2 implements the primary innovation of Wormhole Mechanics. To reward long term miners each Wormhole Event occurs randomly during each epoch and is 180 blocks long; with a reward of 2970 EMC2 per block. [33]\\

%The EMC2 coin is released by the Einsteinium Foundation. The Einsteinium Foundation exists to raise money to help fund cutting edge scientific projects. They launched Einsteinium, March 1st 2017, to help them realize their goal of funding cutting edge science.\\

Einsteinium coin uses a Proof of Work scrypt algorithm and will have a total of 299 million coins. 2.5\% of each block will go to the Einsteinium Foundation with 2\% to be given to science projects and .5\% going towards faucets, give-aways and marketing. Einsteinium had a good launch and there was no premine. [33]\\
 
Einsteinium is therefore already implementing what it is described in the chapter about Gridcoin funded science but does not reward users for performing BOINC computations.
