\subsection{Gridcoin Funded Science}

Although the scientific method is the cornerstone of modern society, it has also some dark sides or at least it can be further ameliorated. 

The advent of Internet did a lot to increase communication between scientists. However, it also introduced the problem of plagiarism and amplified the problem of falsified data [48]. For the increasing amount of papers, there is not enough incentive and competent people to guarantee an independent and competent peer review process [49,50,51]. Sometimes, papers are not available for free but only through expensive subscriptions (remember Aaron Swartz [52]). It is possible to read a paper, but seldomly the source code of software in the paper is made available to the public. Therefore it is very difficult to independently test what it is written in the paper, unless a considerable amount of work is invested to replicate the software. Datasets are often kept secret to discourage competitors, although most of the time they are put together with money from taxpayers. The significance problem [53] hiddenly points to another big scientific issue: it is quite easy to collect money for mastodontic projects inline with the mainstream of science thinking, but it is very difficult to get even little funding to test an idea which is outside of mainstream.\\

If Gridcoin would introduce a fixed amount in the coinbase of each block with empty input and output a special Gridcoin address named 'Gridcoin Funding', the network would collect Gridcoins to that special address for each block added to the blockchain. People with an idea who would like to get funded, they would first submit their proposal in form of a whitepaper to the Gridcoin community plus a Gridcoin address to receive fundings plus the amount of Gridcoins needed to fullfill the project. If the proposal fits some basic prerequisites, a special Gridcoin poll will be created on the blockchain asking the community to get funds for the project. In this special poll the Gridcoin address of the project and the requested sum of Gridcoins will be hardcoded. If the community approves the poll, funds will be automatically sent to that Gridcoin address.\\

Getting all fundings in the beginning are normally a bad motivator. So there will be a mechanism which will pay out the amount to the Gridcoin address split in fixed intervals, for example monthly. There will be a mechanism to issue a second poll to ask to stop of fundings, in case the project is not performing as expected.\\

On September 10, 2017, the market capitalization of Bitcoin was 67'641'887'163 \$ composed by a circulating supply of 16'556'575 of Bitcoins at a price of 4'085.50 \$. If one percent of that Bitcoins would have been spent to a similar fund described above, the fund would have about 676 million dollars available for projects.\\

Malaria is a diffuse disease in Africa but pharmaceutical companies are not investing in medecines for it, not because they are bad as in any good conspiracy theory, but simply because they can not afford to pay the research and development bill with the money they would collect from poor people with that disease. Imagine for a moment a pharmaceutical company asking for a 100 million dollars from the Gridcoin fund to start research on malaria cures. Although utopic, the scenario is not completely unthinkable, viewed the numbers of Bitcoin in the previous paragraph.\\

Imagine Elon Musk funding the settlement of humans on Mars with Gridcoins or an X-Wing spaceship lurking in your garage bought second hand and developed with Gridcoin funding.\\

Having traditionally funded science compete against Gridcoin funded science could spark the next scientific revolution since the age of enlightenment.\\
