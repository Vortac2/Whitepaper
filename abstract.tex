\begin{abstract}
Gridcoin [6] is a decentralized, open source cryptocurrency which performs transactions without the need for a central issuing authority. Gridcoin securely rewards volunteer distributed computing performed on the BOINC [5] platform. BOINC is a volunteer computing grid which combines the processing power of individual users for the purposes of scientific research. The platform is already home to numerous independent projects with such diverse goals as curing diseases like cancer, AIDS, Ebola and Zika, assessing climate change, perfecting orbits of potential hazardeous asteroids and many others [11-20]. Since its creation in the early 2000s, the BOINC network has maintained an active userbase consisting of hundreds of thousands of participants from around the globe. Gridcoin was created in part to incentivize and further grow the BOINC userbase by rewarding calculations performed on the network.\\

Though heavily based on Bitcoin, Gridcoin distinguishes itself in its "environmentally friendly" method of securing the network. Instead of using Proof of Work, which hinges on the wasteful task of inverting a hash function, Gridcoin implements the novel approach of Proof of Stake [2] linked to Distributed Proof of Research [3], developed specifically for Gridcoin. In particular, Proof of Research replaces hash function inversion, which has no value outside of the network which it secures, with useful scientific computation on the BOINC network. Consistent with Gridcoin's decentralized structure, selection of which scientific projects to include in Proof of Research is carried out using a voting mechanism embedded in the blockchain. Additionally, we propose to reserve a small percentage of the minted Gridcoins in a special fund to sponsor scientific projects that may fall outside the scope of government and industry funded research. This blockchain-based funding mechanism ushers in the new and exciting prospect of scientific research directed by the goals and interests of a decentralized international community. 
\end{abstract}


