\section{Appendix}

\subsection{Terminology}

\begin{itemize}
  \item \textit{51\% attack} : A 51\% attack is a situation where more than half of the computing power on a network is operated by a single individual or concentrated group, which   gives them complete and total control over a network. Things that an entity with 51\% of the computing power can do include, but are not limited to: [39]\\
  \begin{itemize}
	\item Halting all mining.
	\item Halting and manipulating all interpersonal transactions.
	\item Use singular coins over and over.
  \end{itemize}
  \item \textit{bitcoin} : The first digital currency able to decentralize the account's ledger by means of a blockchain. 
  \item \textit{beacon} : A beacon is a particular network transaction containing the CPID of a user to signal the network that this user is starting to calculate valid scientific research.
  \item \textit{block} : Blocks are essentially pages in a ledger or record keeping book. Blocks are the files where unalterable data related to the network is permanently stored. Forever. Like eternity.
  \item \textit{blockchain} : A blockchain is a data system that allows for the creation of a digital ledger of transactions on a non-centralized network. Cryptography is the main operator that allows for users to engage with the ledger without the need for any central figurehead. In layman’s terms, this means that people and computers all over work together to create a network instead of a network being made by one single person or company. This network is enabled and protected through cryptography! We have seen this used in currency, data transfer and on. The blockchain is comprised of “blocks” and is constantly growing as each new record, datum, or block is added onto the chain for everyone to see. [39]
  \item \textit{BOINC} : Berkeley Open Infrastructure for Network Computing is a software which allows to donate computational time to scientific research. It is possible to participate to several projects, each run by different and recognized scientific institutions. It is a client/server architecture. Clients calculate tasks in form of work units and report them to servers running the BOINC project software.
  \item \textit{BOINC client} : A BOINC client runs on a users's computer, mini-PC (e.g. Raspberry Pi), cellphone, tablet, or any other device with a CPU (watches might come in the future). It downloads work units from a BOINC server and processes them. When results are available, it reports them back to the BOINC server.
  \item \textit{BOINC server} : A BOINC server runs a particular BOINC project. It distributes work unit to BOINC clients and collects the submitted results, once the workunits are calcuated by the client.
  \item \textit{BOINC project} : A BOINC project is an umbrella for several tasks which are distributed by a recognized scientific institution. A BOINC project publishes scientific results from time to time after enough work units are processed.
  \item \textit{cobblestone} : A cobblestone corresponds to 432 billion floating point operations, 432 gigaflops or 0.432 teraflops.
  \item \textit{consensus} : Consensus is agreement achieved over a peer to peer network when all network nodes agree on something, for example a transaction of a digital currency between two users or a reward given to a particular user. It is achieved by cryptographical means. A 51% attack can break it.
  \item \textit{CPID} : A Cross Project Identifier is assigned to each node running BOINC and it is used to uniquely identify the node both in the gridcoin network and in the BOINC infrastructure.
  \item \textit{cryptocurrency} : A digital currency in which encryption techniques are used to regulate the generation of units of currency and verify the transfer of funds, operating independently of a central bank. [41]
  \item \textit{ethereum} : The first digital currency to introduce a flexible blockchain by means of a Turing complete virtual machine.
  \item \textit{Gridcoin} : The first digital currency that rewards users for performing meaningful scientific research.
  \item \textit{Kimoto's Gravity Well} : It is a common method for difficulty readjustment in cryptocurrency mining. 
  It was first implemented by Megacoin and named after that coin’s lead developer. It was designed to solve problems caused by multipool mining.\\
  Multipools are mining collectives which automatically switch to mining whichever cryptocurrency is the most profitable. The problem with this is that when a multipool targets a coin the increase in hashing power makes the difficulty soar, which in turn makes profitability of mining crash. The multipool then moves on to the next target and the network is left struggling to find miners to keep maintaining the network with the new high rate of difficult. Kimoto’s Gravity Well algorithm allows difficulty to be readjusted every block, meaning that it can respond to both increases and decreases in hashing power immediately and keep the difficulty level at an appropriate level. [46]
  \item \textit{magnitude}: The magnitude in the BOINC network is the ratio between the recent work done by a single node against the work done by the entire BOINC network done on a particular project.
  \item \textit{"Neural network"} : The Gridcoin neural network is a computation validated by the nodes on the network that decides how many gridcoins each node receives in exchange for the scientific research performed. It is not to be confused with a neural network in the artificial intelligence context.
  \item \textit{peercoin} : The first digital currency which introduced Proof of Stake to avert the huge energy consumption of Proof of Work.
  \item \textit{Proof of Work (PoW)} : Proof of work was a concept originally designed to sieve spam emails and prevent DDOS attacks. A Proof of Work is essentially a datum that is very costly to produce in terms of time and resources, but can be very simply verified by another party. The proof of work for Bitcoin is referred to as a “nonce,” or number used only once. This has been considered an energy intensive alternative to proof of stake as the computers unfortunately have to be on and running, which also drives the market towards centralization of hashing power… which is what the blockchain aims to defeat! [39]
  \item \textit{Proof of Stake (PoS)} : Proof of stake has been considered the greener alternative to PoW. Where PoW requires the prover to perform a certain amount of computational work, a proof of stake system requires the prover to show ownership of a certain amount of money, or stake. [39]
  \item \textit{Proof of Research (PoR)} : Proof of Research is the novel approach introduced by the Gridcoin network to reward users performing scientific calculations to the benefit of mankind.
  \item \textit{Recently Averaged Credit (RAC)} : Recently Averaged Credit is a function that calculates the amount of scientific work performed starting from the work units reported by a user to a BOINC project server.
  \item \textit{Research Saving Account (RSA)} : this term denotes research performed by a user which was not rewarded with gridcoins yet.
  \item \textit{Superblock} : A superblock is a specificity of the Gridcoin network. It is a blockchain block that contains how much block which was done by each single user.
  \item \textit{transaction} : A transaction is a transfer of a cryptocurrency value that is broadcasted to the network and collected into blocks. A transaction typically references previous transaction outputs as new transaction inputs and dedicates all input Bitcoin values to new outputs. Transactions are not encrypted, so it is possible to browse and view every transaction ever collected into a block. Once transactions are buried under enough confirmations they can be considered irreversible.\\
Standard transaction outputs nominate addresses, and the redemption of any future inputs requires a relevant signature.\\
All transactions are visible in the block chain, and can be viewed with a hex editor. A block chain browser is a site where every transaction included within the block chain can be viewed in human-readable terms. This is useful for seeing the technical details of transactions in action and for verifying payments. [40]
  \item \textit{workunit} : A single scientific calculation performed on a BOINC client. A work unit can last from minutes (typically if it is run on a modern graphic card, a GPU) 
to entire months (some climate simulation can last several weeks).
\end{itemize}

\subsection{Gridcoin Whitelist}

The Gridcoin Whitelist [42] includes all BOINC projects which give a reward in Gridcoin. It is compiled by people's vote persisted on the blockchain. At time of writing (October 2017) the following project are whitelisted:\\

\begin{itemize}
	\item {\em Amicable Numbers} : Amicable Numbers is an independent research project that uses Internet-connected computers to find new amicable pairs. Currently searching the 10\^20 range.	Private	
	\item {\em Asteroids@home} : Asteroid research - it uses photometric measurements of asteroids observed by professional big all-sky surveys as well as 'backyard' astronomers. The data is processed using the lightcurve inversion method and a 3D shape model of an asteroid together with the rotation period and the direction of the spin axis are derived.	Charles University in Prague
	
	\item {\em Citizen Science Grid} : Umbrella project for DNA@Home, SubsSet Sum@Home, and Wildlife@Home. University of North Dakota
	\item {\em ClimatePrediction.net} : Run computer models to simulate the climate for the next century, producing predictions of temperature, rainfall and the probability of extreme weather events. 	University of Oxford. Additional note: due to delays in reporting user's statistics, this project is not in the whitelist, but will be as soon as the issue is solved.
	\item {\em Collatz Conjecture} : Attempting to disprove the Collatz Conjecture.	Private		
	\item {\em Cosmology@Home} : Darkmatter/Universe Model Research	Department of Astronomy at the University of Illinois at Urbana-Champaign
	\item {\em DrugDiscovery@Home} : Discovery of new drugs for the most dangerous and widespread diseases.	Digital BioPharm Ltd
	\item {\em Einstein@home} : Search for spinning neutron (pulsars) stars using data from the LIGO gravitational-wave detectors, the Arecibo radio telescope, and the Fermi gamma-ray satellite	Univ. of Wisconsin - Milwaukee, Max Planck Institute
	\item {\em Enigma@Home} : The M4 Project is an effort to break 3 original Enigma messages with the help of distributed computing. The signals were intercepted in the North Atlantic in 1942 and are believed to be unbroken.	Private	
	\item {\em GPUgrid} : Full-atom molecular simulations of proteins	Private Sponsors
	\item {\em LHC@Home} : Accelerator Physics. CERN in Geneva, Switzerland
	\item {\em Milkyway@home} : Creation of a 3D map of the Milky Way galaxy using data gathered by the Sloan Digital Sky Survey. This project enables research in both astroinformatics and computer science.	Rensselaer Polytechnic Institute
	\item {\em Moo! Wrapper} : Moo! Wrapper brings together BOINC volunteer computing network resources and the Distributed.net projects. It allows a BOINC Client to participate in the RC5-72 challenge.	Distributed.Net	
	\item {\em NFS@Home} : Lattice sieving step in Number Field Sieve factorization of large integers. Many public key algorithms, including the RSA algorithm, rely on the fact that the publicly available modulus cannot be factored. If it is factored, the private key can be easily calculated.	California State University Fullerton	
	\item {\em NumberFields@home} : Research in number theory. Number theorists can mine the data for interesting patterns to help them formulate conjectures about number fields.	Arizona State University, school of Mathematics	
	\item {\em PrimeGrid} : Search for prime numbers. Primes play a central role in the cryptographic systems which are used for computer security. Through the study of prime numbers it can be shown whether current security schemes are sufficiently secure.	Private, supported by Rackspace
	\item {\em Rosetta@Home} : Protein structure prediction that may ultimately lead to finding cures for some major human diseases.	University of Washington
	\item {\em SETI@home} : Search for Extraterrestrial Intelligence (SETI). University of California, Berkeley
	\item {\em SRBase} : Attempting to solve Sierpinski / Riesel Bases up to 1030.	The project is in collaboration with the Mersenne CRUS project.	
	\item {\em theSkyNet POGS} : Astronomy Research. Combine the spectral coverage of GALEX, Pan-STARRS1, and WISE to generate a multi-wavelength UV-optical-NIR galaxy atlas for the nearby Universe. Calculate physical parameters such as: star formation rate, stellar mass of the galaxy, dust attenuation, and total dust mass of a galaxy; on a pixel-by-pixel basis using spectral energy distribution fitting techniques	International Centre for Radio Astronomy Research (ICRAR), a joint venture of Curtin University and The University of Western Australia	
	\item {\em TN-Grid} : The gene@home project is an implementation of the PC-IM algorithm, whose purpose is to expand Gene Regulatory Networks (GRN). Each network is a graph that specifies the causal relationships inside this set of genes, and helps in studying the gene expression phenomenon: the process through which the DNA is transcribed into RNA and the RNA translated into proteins.	National Research Council of Italy (CNR)
	\item {\em VGTU} : Distributed computing platform for scientists of Vilnius Gediminas Technical University (VGTU).	Vilnius Gediminas Technical University
	\item {\em Universe@home} : Physics and Astronomy	University of Warsaw
	\item {\em YAFU} : Factorize numbers of 70-130 digit length which are needed to bring Aliquot Sequences to a size of 130.	Private	
	\item {\em yoyo@home} : Brings existing distributed computing projects to the BOINC world using the BOINC Wrapper technology	Supported by Rechenkraft.net 
	\item {\em World Community Grid} : FightAIDS@home, Smash Childhood Cancer, fight Tuberculosis, research better materials for solar panels.	Sponsored by the IBM responsibility initiative
	
\end{itemize}
		

