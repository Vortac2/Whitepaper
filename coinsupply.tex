\subsection{Coin Supply and Security Measures}

With a fixed daily research coin supply per project ($\Theta = G/n$) it would not be ensured that the inflation rate is always the same; it could vary depending on how many new researchers join the project \textit{p}. Because of this, the amount of average payouts over the last 14 days is used as a lagging indicator of how much was paid out recently - if very little was paid out in the last 14 days more is paid out now and the other way around.

\[ G = MaxDailyEmissions - AvgDailyPaymentsPaidInLast14Days \]

There are also a few security rules. For example time since last payment in days ($\tau$) cannot be greater than 6 months, otherwise there is no payment and the coins paid out per user does have a very high upper limit (~5000) per stake.
The $MaxDailyEmissions$ is set to 50000 Gridcoins, which at the current coin supply means an research driven inflation of around 5\%. This rate however will grow smaller, as the coin supply grows but the amount of coins produced per day stays the same. Additionally the $inflationreward$ is chosen, so that the interest inflation is around 1.5\% per year.
