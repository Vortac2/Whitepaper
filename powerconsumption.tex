\subsection{Gridcoin Power Consumption Estimate}

We start to calculate the consumption of Gridcoin's closest relative, Bitcoin. 
We use first the approach explained in [29], where it is assumed that an ASIC miner (a dedicated hash processor) calculates with 1 Watt of power one Gigahash/s. According to [26], Bitcoin speed on August 26, 2017 was 6'354'668.57 terahash/s or 6'354'668'570 gigahash/s or 6'354'668'570 W. These are 6.354 GW of power. They are equivalent to the power produced by 6 large nuclear power plants. Over one year, which has 365 $\cdot$ 24 hours (=8760 hours), we get 6.354 GW $\cdot$ 8760h = 55'661 GWh or 55.661 TWh/year.\\ 

By comparison, the Swiss Federal Railways consume about 3 TWh/year, and CERN in Geneva 1TWh/year.

Another approach outlined in [27] for the Bitcoin Energy Consumption Index gives 16.2 TWh/year as consumption estimate for last year. Steps to calculate the mentioned index are:\\

\begin{itemize}
	\item Calculate total mining revenues
	\item Estimate what part is spent in electricity
	\item Find out how much miners pay per kWh
	\item Convert costs to consumption
\end{itemize}

The Bitcoin Energy Consumption Index assumes that miners will ultimately spend 60\% of their revenues on operational costs on average. For every 5 cents that were spent on operational costs it is assumed that 1 kilowatt-hour (kWh) was consumed. [27]\\

Price movements can be small or large, but new energy-hungry machines won't all appear overnight. Realistic behaviour is introduced by linking price dynamics to the expected time required for producers to fully respond to a changing situation. [27]

Ethereum, another common cryptocurrency consumes with the above method 5.5 TWh of power in one year [28].\\

To calculate power consumption of BOINC on September 2, 2017, we choose the approach outlined in [21]. We look first at the daily cobblestones added in this day: 3'300'282'122 cobblestones. The subset of Gridcoin users did 489'559'755. These numbers roughly compare to the numbers used in the previous chapter to estimate BOINC and Gridcoin speed, taken on August, 22, 2017.\\ 

We now take following assumption: the BOINC average user owns a standard CPU like an Intel Core i5 with an average GPU like Nvidia GTX 1060. The Intel Core i5 features 4 cores times 4.57 GFlops = 18.28 GFlops with 77W consumption and the 1060 graphic card does 3'850 GFlops with 120W of power. This system totally features 3868.8 GFlops: 0.47% is done by the CPU and 99.53% by the GPU \\

By diving the cobblestones by 200 we get first the speed in GFlops on September 2, 2017: 16'501'410 GFlops for BOINC and 2'447'799 GFlops for Gridcoin. Power calculation of BOINC becomes:

\[ 16'501'410 \cdot 0.0047 \cdot 77/18.28 + 16'501'410 \ cdot 0.9953 \cdot 120/3850 = 838'601 W = 0.839 MW \]
\[ 0.838 MW \cdot 24 h \cdot 365 days = 7'350 MWh = 7.350 GWh \]

0.838 MW can be compared to the power produced by a little hydro power plant in the Alps. \\
In one year, BOINC consumes about 7.350 GWh. Similarly, Gridcoin power calculation gets:

\[ 2'447'799 \cdot 0.0047 \cdot 77/18.28 + 2'447'799 \ cdot 0.9953 \cdot 120/3850 = 124'397 W = 0.124 MW \]
\[ 0.124 MW \cdot 24 h \cdot 365 days = 1'086 MWh = 1.086 GWh \]

This calculation assumes that most power to run Gridcoin is in Proof of Research and that running the client is almost negligible compared to the PoR part.\\
If we assume that one household roughly consumes 5 kW, then Gridcoin power consumption corresponds to 25 households.\\


