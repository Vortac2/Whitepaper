\section{Rationale about investing in Gridcoin}

The current Bitcoin price surge (October 2017) has similarities to the dotcom bubble in the beginning of the millenium. The dotcom bubble created as side effect all Internet services we are using today.\\

The reason why there is high demand of Bitcoin is threefold:\\

\begin{itemize}
  	\item The Bitcoin supply is limited. The embedded mining algorithm of Bitcoin halves the bitcoin reward for any new block in the blockchain every 4 years and stops creating Bitcoins when 20.67 million bitcoins are created. By 2021 most existing bitcoins in circulation will be minted.
           \item Bitcoin is currently used by many players, including big investors, to buy other minor cryptocurrencies and to fund new companies that use cryptocurrencies as crowdfunding mechanism.
           \item Bitcoin is seen by many as digitalized gold and many put their money into bitcoin as long term investment.  
\end{itemize}

Bitcoin has two main drawbacks: we saw in the previous chapter that Bitcoin can not be used as mean to exchange goods between people as its transaction speed is too low. We also saw that Bitcoin is not enviromentally susteinable: the algorithm which protects and verifies transactions and creates new blocks is consuming too much power.\\

It is reasonable to think that soon or later the regulator will prohibit Bitcoin due to enviromental concerns or at least that the mindset of people will turn to other more eco-friendly cryptocurrencies, especially as the consequences of climate change will be the more visible.\\

Gridcoin finds a way to consume the energy wasted in Bitcoin to the benefit of mankind: the energy used to secure the blockchain of Gridcoin is also used to advance scientific research in fields where a great amount of computation is needed.\\

For the sake of completeness we mention here another ecofriendly cryptocurrency with a bright future: Solarcoin [38]. Solarcoin is a Proof of Stake cryptocurrency where little energy is needed to secure the blockchain. Additionally, people running solar installations on their roofs are awarded one solarcoin for each megawatthour their solar installation produce. Awards are retroactive, but need to be proved by documents verified by the grid responsibles and by a non password protected website showing the current production of the solar installation.\\